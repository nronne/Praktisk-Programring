%\pdfoutput=0
\documentclass[twocolumn]{article}
\usepackage{graphicx}
\title{An example of using \LaTeX{} and Gnuplot: the exponential function}
\author{D.V.~Fedorov}
\begin{document}
\maketitle

\noindent
The exponential function can be defined as the solution to the
differential equaion,
\begin{equation}\label{diff-eq}
y'(x)=y(x) \;,
\end{equation}
with the boundary condition
\begin{equation}\label{bound-cond}
y(0)=1 \;.
\end{equation}

The argument can be reduced to $x\in[0,1]$ using identities,
\begin{eqnarray}
\exp(x)&=&\exp\left(\frac{x}{2}\right)\exp\left(\frac{x}{2}\right) \;, \\
\exp(-x)&=&\frac{1}{\exp(x)} \;.
\end{eqnarray}

Once the argument is reduced, the differential equation~(\ref{diff-eq})
can be integrated numerically with sufficient accuracy.

\begin{figure}[h]
\input{plot-exp.tex}
\caption{Illustration of the exponential function.}
\end{figure}

\end{document}
